\documentclass{article}
\usepackage[utf8]{inputenc}
\usepackage{blindtext}
\usepackage{multicol}

\title{Serverless HPC Over Cloud}
\author{Davit Petrosyan}
\date{July 5}

\begin{document}

\maketitle

\begin{abstract}
  While cloud technologies are getting more power, the High-Performance Computing technologies are upgrading not fast enough. Thus, the need of modern solution for running HPC workloads over cloud remains crucial. 
  
  We aim to describe an architecture, that enables one of the most important cloud capabilities for HPC workloads - running HPC workloads server-less over cloud (Shoc). Utilizing abstraction power of container technologies like Docker and combining with scheduling capabilities of Kubernetes, Shoc allows running any both CPU-intensive and Data-intensive workloads in the cloud without managing infrastructure ourselves. 
  
\end{abstract}

\section{Introduction}
One of the most important and powerful technologies in the world of high-performance computing is MPI. We will focus on MPI programs, however, the architecture proposed in the paper will be capable of running workloads based on MPI (OpenMPI), Spark with Hadoop, CUDA-enabled applications and more. 
While running MPI job on a single machine is a trivial task, it becomes harder when workload should run on multiple machines connected within a network, usually a high-speed one. Usually, once communication mechanism is chosen and correctly set-up via, say, SSH, one can simply run distributed workload and utilize required resources all over the nodes in the cluster.

Next, when it comes to workload orchestration, one will use a resource management and scheduling system to be able to queue tasks and allocate resources within given requirements and availability. This is, however, more than important especially when resources are limited and tasks are resource-intensive. So, based on above, it is clear that managing all this together without proper infrastructure is not an easy task. Usually it requires lots of manual effort to set this up even on a cluster with few nodes.  




\end{document}


